\chapterquote{%
To see what is general in what is particular and\\
what is permanent in what is transitory is\\[.4ex]
the aim of scientific thought.}
{-- Alfred North Whitehead, \textit{An Introduction to Mathematics} (1911)}

Recent advances on wireless communications systems are pointing towards heterogeneous networks solutions with dense deployment of nodes \cite{hossain2014evolution}.
%
These wireless networks are expected to provide reliable connectivity to high-mobility devices, to users with millisecond delay constraints, and applications with gigabits per second of data rate requirements \cite{peng2015recent}.
%
This unceasing demand for higher throughput and lower latency leads to an ever-increasing traffic of packets.
%
Thus, the performance of wireless networks tends to be, each day more, limited by interference. % which depends on the spatial distribution of the interferers, the arrival process of packets, the queueing dynamics.

In this thesis, the studied telecommunication problem is the analysis of spatio-temporal traffic on wireless networks limited by interference.
%
In this first chapter, we provide a simple and introductory system model along with the approach for the analysis and the studied scenarios.
%
Note that a rigorous, general, and detailed model is provided in Chapter~\ref{cap:P2_00}.

\newpage
%-%-%-%-%-%-%-%-%-%-%-%-%-%-%-%-%-%
\section{The Problem}

In a wireless network, we have a set of transmitters and a set of receivers located across the space.
%
As time passes, for each transmitter is given packets that must be transmitted to a determined receiver as soon as possible.
%
Each packet is stored in a buffer and discarded upon successful reception by the corresponding receiver.

When a transmitter sends a packet, it interferes with other transmissions, and vice-versa. This aggregate interference is determined according to the transmit power of each transmitter, the transmission distance, and some random effects (e.g. fading).
%
Then, we determine whether a transmission was successful depending on the signal and the aggregate interference received by the receiver.

\textbf{What are the \textit{givens} of the problem?}%
\setlist{nolistsep}%
\begin{enumerate}[itemsep=1pt]
    \item The spatial position distribution of the transmitters and receivers.
    
    \item The arrival process of packets for each transmitter.
    
    \item The propagation model that affects the transmitted signals, e.g. path loss, fading.
    
    \item The network operation model or queueing system model, that describes the rules of packet buffering and transmitting, e.g. Aloha, CSMA.
    
    \item The transmission success probability model that decides whether a packet is successfully received based on the received signal and the received interference.
\end{enumerate}

\textbf{What do we want to \textit{know}?}%
\setlist{nolistsep}%
\begin{enumerate}[itemsep=1pt]
    \item Reliability metrics, i.e., metrics related to the transmission and packet success probabilities.

    \item Throughput metrics, i.e., metrics related to the number of successfully delivered packets, such as mean throughput, stability.
    
    \item Delay metrics, i.e., metrics related to the amount of time to deliver packets, such as mean delay, queue load, age-of-information.
\end{enumerate}

Regarding the \textit{givens}, we can relax them in some aspects.
%
For example, instead of working with a specific path loss model, we let the path loss model be an arbitrary function with some constraints.
%
When the problem is analytically tractable, usually there is little cost to let some well-chosen parameters (or even functions) be arbitrary.
%
Then, once we let a set of \textit{givens} arbitrary, we attain a better comprehension of the relationship between these \textit{givens} and the \textit{knowns} we are interested in.
%
We also obtain a generalization of the problem.

% In this sense, our approach is to study the tractable models used by the scientific community and let some of the ``givens'' be arbitrary.
% %
% More specifically, we choose the ``givens'' (e.g. arrival rate, access probability) for which we are interested in better understanding their relationship with the ``knowns'' (e.g. throughput, delay).

More specifically, the standard \textit{given} scenario of this thesis assumes a single class of users sharing the same frequency bandwidth and the following \textit{givens}:
\begin{enumerate}
    \item High-mobility Poisson network, i.e., the nodes' positions follow a Poisson point process.
    \item Packets arrive according to a Poisson (or Bernoulli) process.
    \item The signals are subjected to Rayleigh fading and power law path loss model.
    \item Packets are queued in infinite buffers, and transmitters access the channel according to a Poisson (or Bernoulli) process.
    \item Transmission success is decided by the capture model.
\end{enumerate}

% a high-mobility Poisson network under random-access, limited by interference, subject to Rayleigh fading and power law path loss model, packets are queued in infinite buffers, transmission success is decided by the capture model, and there is a single class of users sharing the same frequency bandwidth.
%
This standard model from the literature is easily (analytically) tractable, as it is shown in Section~\ref{sec:P1_03_02}.
%
For each scenario studied in this thesis, we list in Table~\ref{tab:givens} the \textit{givens} that we modified, flexibilized and generalized from the standard model, and the \textit{knowns} for which we established a connection.

\begin{table}[htb]
    \centering
    \begin{tabular}{c|l|l}
    \hline
    Location & Change on the \textit{givens}  & The \textit{knowns} \\
        \hline\hline
    Ch.~\ref{cap:P2_01} & transmission success models, no buffer
        & throughput, reliability \begin{tabular}[c]{@{}l@{}}~\\ ~ \end{tabular}\\\hline
    Ch.~\ref{cap:P2_02} & \begin{tabular}[c]{@{}l@{}}maximum number of retransmissions,\\ general transmission success model \end{tabular}
        & \begin{tabular}[c]{@{}l@{}}throughput, reliability,\\ delay, age-of-information\end{tabular}\\\hline
    Sec.~\ref{sec:high-mobility} & $N$ classes of users
        & \begin{tabular}[c]{@{}l@{}}stability, throughput, reliability,\\ delay, queue load\end{tabular} \\\hline
    Sec.~\ref{sec:bandwidth} & $2$ classes of users, bandwidth partition
        & \begin{tabular}[c]{@{}l@{}}stability, throughput, reliability,\\ delay, queue load\end{tabular}\\\hline
    Ch.~\ref{cap:P2_04} & \begin{tabular}[c]{@{}l@{}}static elements, general path loss model,\\ general distribution of link distances\end{tabular}
        & \begin{tabular}[c]{@{}l@{}}stability, queue load,\\
            stationary distribution\end{tabular}\\
    \hline
    \end{tabular}
    \caption{Variations on the \textit{givens} of the base model.}
    \label{tab:givens}
\end{table}

A thorough review of the literature models used in this thesis and their applications can be found in the tutorials \cite{haenggi2021stochastic, cardieri2010modeling}.

%-%-%-%-%-%-%-%-%-%-%-%-%-%-%-%-%-%
\section{Outline of the Thesis}

We assume the reader is familiar with probability theory, calculus, and optimization.
%
However, we do not assume familiarity with measure theory nor stochastic geometry. That is why we provide a succinct review on these subjects.
%
This knowledge is necessary for stating and understanding definitions and theorems from this thesis and from the well-celebrated books of stochastic geometry and wireless networks \cite{baccelli2010stochastic} and \cite{haenggi2012stochastic}.
%
Nevertheless, the reader that is not interested in mathematical details and proofs can skip some parts as indicated throughout the manuscript in notes highlighted with light-blue boxes.
%
Furthermore, for ease of reading, on the left side of every proof, there is a vertical line. Therefore, the proofs that do not interest the reader can be easily skipped.

In this manuscript, we differentiate \textit{lemmas}, \textit{theorems}, and \textit{propositions} in the following manner.
%
We let the short and mathematical results that are used to prove other statements be the \textit{lemmas}.
%
For the results which are related to the physical problem, yet they have a mathematical basis in the sense that their existence is independent of the physical problem, those we call \textit{theorems}.
%
Finally, the results which are strongly related to the physical problem and alone do not have significance, those we call \textit{propositions}.

Throughout the thesis, we also have the \textit{remarks} and the \textit{notes}, which are highlighted in gray and light-blue boxes, respectively.
%
Their purpose is to comment on aforementioned results.
%
We reserve the more contextualized and formal comments to the \textit{remarks}, and the more digressive and informal comments to the \textit{notes}.

% We decided to include several definitions, because when we say that the system model is over a ``stationary and ergodic Poisson network'' the reader is able to check what exactly that means.

% \begin{itemize}
%     \item Mathematical optimization.
%     %
%     This knowledge is necessary for stating and understanding definitions and theorems from stochastic geometry presented on the well celebrated books \cite{baccelli2010stochastic} and \cite{haenggi2012stochastic}.
    
%     \item (alternatively) In this thesis we need measure theory (applied to integration and probability) to present and prove some results. Since this is not a common subject in the engineering curriculum, we provide a succinct introduction of this topic in Chapter~\ref{cap:P1_02} focusing on the results we need.
    
%     \item The thesis is readable without measure theory, except for Chapter~\ref{cap:P1_03} and some proofs.
    
%     \item We tried to be mathematically rigorous because we have general results and it is not possible to simulate every representative choice of model to guarantee them;
    
%     \item General results in the sense that they are valid for arbitrary functions that represent, e.g, the path loss, the success transmission model, the link distance distribution.
    
%     \item We are following the formalism adopted in the well celebrated books in the area of stochastic geometry applied to wireless networks \cite{baccelli2010stochastic}, \cite{haenggi2012stochastic}.
% \end{itemize}

This thesis is organized in nine chapters and two appendices.
%
This first chapter presents, in an intuitive and general manner, the main problem and the approach adopted to study it.
%
The other chapters are summarized below.

\begin{description}
    \item[Chapter~2:] We present the notations and definitions used throughout the manuscript. We also provide a succinct review of measure theory applied to integration and probability. They are necessary to understand stochastic geometry and several proofs of this thesis.
    
    \item[Chapter~3:] In this chapter, we present the theory of point processes and random measures. Then, we present some known and extensively used results of stochastic geometry applied to wireless networks.
    
    \item[Chapter~4:] This is the first chapter with contributions.
    We present a general system model for a wireless network and some definitions, which are used until the end of the thesis.
    
    \item[Chapter~5:] We consider a Poisson process to model packet transmissions and characterize the distribution of the interference in this model.
    
    \item[Chapter~6:] This chapter includes retransmission and analyzes the impact of this strategy in the throughput and delay.
    
    \item[Chapter~7:] Here we include the spatial positions of the nodes in the analysis. We also model the buffers of the transmitters as queues and study their stability.
    
    \item[Chapter~8:] In this chapter we introduce static elements. More precisely, the link distances between transmitters and receivers are fixed across time. This consists of a challenging class of problems because static elements hinder tractability.
    
    \item[Chapter~9:] We conclude the thesis listing the main contributions and prospects for future research.
    
    \item[Appendix~A:] Some definitions and results of real analysis used in the thesis.
    
    \item[Appendix~B:] A brief review of queueing theory (delay, stability).
\end{description}

%-%-%-%-%-%-%-%-%-%-%-%-%-%-%-%-%-%
\section{Author's Publications}

This thesis is based on three published international journal papers \cite{dester2018,dester2021unique, dester2021retrans}, and one submitted paper \cite{dester2021part}.
%
Besides, the author has three published international journal papers \cite{dester2017stationary, francisco2020, bergamo2021} and two conference papers \cite{dester2018mtc, helder2019lora} that are not covered by this thesis.
