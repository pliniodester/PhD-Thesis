%ARQUIVO DE PREAMBULO DA TESE - PACOTES E CONFIGURA\c{C}\~{O}ES

\documentclass[
	% -- op\c{c}\~{o}es da classe memoir --
	12pt,				% tamanho da fonte
    % 	openright,			% cap\'{\i}tulos come\c{c}am em p\'{a}g \'{\i}mpar (insere p\'{a}gina vazia caso preciso)
	oneside,			% para impress\~{a}o em verso e anverso. Oposto a oneside/twoside
	letterpaper,		% tamanho do papel.
	% -- op\c{c}\~{o}es da classe abntex2 --
	%chapter=TITLE,		% t\'{\i}tulos de cap\'{\i}tulos convertidos em letras mai\'{u}sculas
	%section=TITLE,		% t\'{\i}tulos de se\c{c}\~{o}es convertidos em letras mai\'{u}sculas
	%subsection=TITLE,	% t\'{\i}tulos de subse\c{c}\~{o}es convertidos em letras mai\'{u}sculas
	%subsubsection=TITLE,% t\'{\i}tulos de subsubse\c{c}\~{o}es convertidos em letras mai\'{u}sculas
	% -- op\c{c}\~{o}es do pacote babel --
	% brazil,			% idioma adicional para hifeniza\c{c}\~{a}o
	%french,			% idioma adicional para hifeniza\c{c}\~{a}o
	%spanish,			% idioma adicional para hifeniza\c{c}\~{a}o
	english,			% o \'{u}ltimo idioma \'{e} o principal do documento
	sumario=tradicional,
% 	sumario=abnt-6027-2012,
    oldfontcommands,
	]{abntex2}

\renewcommand{\ABNTEXchapterfont}{\fontfamily{cmr}\selectfont}
\renewcommand{\ABNTEXchapterfontsize}{\HUGE}

% ---
% PACOTES
% ---

% ---
% Pacotes fundamentais
% ---
\usepackage{cmap}				% Mapear caracteres especiais no PDF
% \usepackage{lmodern}			% Usa a fonte Latin Modern			
\usepackage[T1]{fontenc}		% Selecao de codigos de fonte.
\usepackage[utf8]{inputenc}		% Codificacao do documento (convers\~{a}o autom\'{a}tica dos acentos)
\usepackage{lastpage}			% Usado pela Ficha catalogr\'{a}fica
\usepackage{indentfirst}		% Indenta o primeiro par\'{a}grafo de cada se\c{c}\~{a}o.
\usepackage{color}				% Controle das cores
\usepackage[pdftex]{graphicx}	% Inclus\~{a}o de gr\'{a}ficos
\usepackage{epstopdf}           % Pacote que converte as figuras em eps para pdf
% \usepackage{lipsum}             % Pacote que gera texto dummy
% \usepackage{blindtext}          % Pacote que gera texto dummy
% ---
		
% ---
% Pacotes adicionais, usados apenas no \^{a}mbito do Modelo Can\^{o}nico do abnteX2
% ---
\usepackage{nomencl}
\usepackage{amssymb,amsfonts,amsthm,amsmath,mathrsfs,bbm,bm,dsfont,etoolbox}
\usepackage[chapter]{algorithm}
\usepackage{algorithmic}
\usepackage{multirow}
\usepackage{rotating}
\usepackage{pdfpages}
% ---


% ---
% Pacotes de cita\c{c}\~{o}es
% ---
\usepackage[brazilian,hyperpageref]{backref}	 % Paginas com as cita\c{c}\~{o}es na bibl
\usepackage[alf,abnt-etal-cite=2,abnt-etal-list=0,abnt-etal-text=emph]{abntex2cite}	% Cita\c{c}\~{o}es padr\~{a}o ABNT

% ---
% Pacote de customiza\c{c}\~{a}o - Unicamp
% ---
\usepackage{unicamp}

% Pacotes adicionais
\usepackage{tikz}
\usetikzlibrary{calc}
\usepackage{pgfplots}
\pgfplotsset{compat=1.7}
\usepgfplotslibrary{fillbetween}
\usetikzlibrary{arrows.meta}
\def\arrowhead{Latex[length=3mm, width=1.5mm]}

\usepackage{caption, subcaption}
\usepackage{enumitem}
\usepackage{empheq}

\usepackage{chngcntr}
\counterwithin{figure}{chapter}
\counterwithin{table}{chapter}

%Options: Sonny, Lenny, Glenn, Conny, Rejne, Bjarne, Bjornstrup
\usepackage[Lenny]{fncychap}
% ---
% CONFIGURA\c{C}\~{O}ES DE PACOTES
% ---
% \def\labelitemi{--} % configura bullets do Itemize
\def\labelitemi{\bf --}

% \patchcmd{\endproof}% <cmd> % coloca linha no fim de proof
%   {\endtrivlist}% <search>
%   {\endtrivlist\par\nobreak\vspace*{\dimexpr-\baselineskip-\parskip}\nobreak\noindent\hrulefill}% <replace>
%   {}{}% <succes><failure>

% ---
% Configura\c{c}\~{o}es do pacote backref
% Usado sem a op\c{c}\~{a}o hyperpageref de backref
\graphicspath{{./eps/}}
\DeclareGraphicsExtensions{.eps}

%customiza\c{c}\~{a}o do negrito em ambientes matem\'{a}ticos
\newcommand{\mb}[1]{\mathbf{#1}}
%customiza\c{c}\~{a}o de teoremas
% \newtheorem{mydef}{Defini\c{c}\~{a}o}[chapter]
% \newtheorem{lemm}{Lema}[chapter]
% \newtheorem{theorem}{Teorema}[chapter]
\floatname{algorithm}{Pseudoc\'{o}digo}
\renewcommand{\listalgorithmname}{Lista de Pseudoc\'{o}digos}


% \renewcommand{\backrefpagesname}{Citado na(s) p\'{a}gina(s):~}
% % Texto padr\~{a}o antes do n\'{u}mero das p\'{a}ginas
% \renewcommand{\backref}{}
% % Define os textos da cita\c{c}\~{a}o
% \renewcommand*{\backrefalt}[4]{
% 	\ifcase #1 %
% 		Nenhuma cita\c{c}\~{a}o no texto.%
% 	\or
% 		Citado na p\'{a}gina #2.%
% 	\else
% 		Citado #1 vezes nas p\'{a}ginas #2.%
% 	\fi}%
% % ---

\renewcommand{\backrefpagesname}{Cited in page(s):~}
% Texto padr\~{a}o antes do n\'{u}mero das p\'{a}ginas
\renewcommand{\backref}{}
% Define os textos da cita\c{c}\~{a}o
\renewcommand*{\backrefalt}[4]{
	\ifcase #1 %
		No citations in the text.%
	\or
		Cited in page #2.%
	\else
		Cited #1 times in pages #2.%
	\fi}%
% ---

% ---
% Configura\c{c}\~{o}es de apar\^{e}ncia do PDF final

% alterando o aspecto da cor azul
\definecolor{blue}{RGB}{41,5,195}

% informa\c{c}\~{o}es do PDF
\makeatletter
\hypersetup{
     	%pagebackref=true,
		pdftitle={\@title},
		pdfauthor={\@author},
    	pdfsubject={\imprimirpreambulo},
	    pdfcreator={LaTeX with abnTeX2},
		pdfkeywords={abnt}{latex}{abntex}{abntex2}{trabalho acad\^{e}mico},
		hidelinks,					% desabilita as bordas dos links
		colorlinks=false,       	% false: boxed links; true: colored links
    	linkcolor=blue,          	% color of internal links
    	citecolor=blue,        		% color of links to bibliography
    	filecolor=magenta,      	% color of file links
		urlcolor=blue,
%		linkbordercolor={1 1 1},	% set to white
		bookmarksdepth=4
}
\makeatother
% ---

% ---
% Espa\c{c}amentos entre linhas e par\'{a}grafos
% ---

% O tamanho do par\'{a}grafo \'{e} dado por:
\setlength{\parindent}{20pt}
% \setlength{\parindent}{2cm}

% Controle do espa\c{c}amento entre um par\'{a}grafo e outro:
\setlength{\parskip}{0.2cm}  % tente tamb\'{e}m \onelineskip

% ---
% Informacoes de dados para CAPA e FOLHA DE ROSTO
% ---
\titulo{Spatio-temporal Traffic in\\[0ex] Interference Networks\\[7ex]
Tráfego Espaço-temporal em\\[4mm] Redes de Interferência}
% \tituloestrangeiro{Tráfego Espaço-temporal em Redes de Interferência}
\autor{Pl\'{i}nio Santini Dester}
\local{Campinas}
\data{2021}
\orientador[\textbf{Orientador:}]{Prof. Dr. Paulo Cardieri}
% \coorientador[Co-orientador]{Prof. Dr. Co-orientador}
\instituicao{%
    Universidade Estadual de Campinas
    \par
    Faculdade de Engenharia El\'{e}trica e de Computa\c{c}\~{a}o
    }
%\tipotrabalho{Tese (Doutorado)}
%% O preambulo deve conter o tipo do trabalho, o objetivo, o nome da institui\c{c}\~{a}o e a \'{a}rea de concentra\c{c}\~{a}o
%\preambulo{Tese apresentada \`{a} Faculdade de Engenharia El\'{e}trica e de Computa\c{c}\~{a}o da Universidade Estadual de Campinas como parte dos requisitos exigidos para a obten\c{c}\~{a}o do t\'{\i}tulo de Doutor em Engenharia El\'{e}trica, na \'{A}rea de Engenharia de Computa\c{c}\~{a}o.}
\tipotrabalho{Tese (Doutorado)}
% \preambulo{Tese apresentada \`{a} Faculdade de Engenharia El\'{e}trica e de Computa\c{c}\~{a}o da Universidade Estadual de Campinas como parte dos requisitos exigidos para a obten\c{c}\~{a}o do t\'{\i}tulo de Doutor em Engenharia El\'{e}trica, na \'{A}rea de \textbf{Telecomunicações}.}
\preambulo{
Doctorate thesis presented to the Postgraduate Programme of
Electrical Engineering of the School of Electrical Engineering of
the State University of Campinas to obtain the Ph.D. degree in
Electrical Engineering, in the field of Telecommunications and
Telematics.\\[2ex]
\textit{Tese de Doutorado apresentada ao Programa de Pós-Graduação
em Engenharia Elétrica da Faculdade de Engenharia Elétrica
e de Computação da Universidade Estadual de Campinas para
obtenção do título de Doutor em Engenharia Elétrica, na área de
Telecomunicações e Telemática.}}
% --- 

\newcommand{\chapterquote}[2]{
  \begin{figure*}[htb]
    \centering
    \begin{tikzpicture}
      \node[text width=12cm,anchor=center] (Q) at (0,0) {\large\textit{#1}};
      \node[gray,anchor=north east] (Ql) at (Q.north west) {\Huge\textbf{``}};
      \node[gray,anchor=south west] (Qr) at ($(Q.south east)+(0,-.34)$) {\Huge\textbf{''}};
      \node[black!80,anchor=north east] (Qa) at (Q.south east) {\small #2};
    \end{tikzpicture}
  \end{figure*}
}
% \newcommand{\chapterquote}[2]{
%   \begin{figure*}[htb]
%     \centering
%     \begin{tikzpicture}
%       \node[text width=12cm,anchor=center] (Q) at (0,0) {\large\textit{#1}};
%       \node[gray,anchor=north east] (Ql) at (Q.north west) {\Huge\textbf{``}};
%       \node[gray,anchor=south west] (Qr) at (Q.south east) {\Huge\textbf{''}};
%       \node[black!80,anchor=north east] (Qa) at (Q.south east) {\small - #2};
%     \end{tikzpicture}
%   \end{figure*}
% }
% \usepackage{fancyhdr}
% \pagestyle{fancy}

\definecolor{mygray}{rgb}{0.95,0.95,0.95}
% \definecolor{myblue}{RGB}{204,229,255}
% \definecolor{salmao}{RGB}{255,204,204}

\newcommand{\red}[1]{\textcolor{red}{#1}}

\DeclareMathOperator*{\argmax}{arg\,max}
\DeclareMathOperator*{\argmin}{arg\,min}

\usepackage{siunitx}

\usepackage[backgroundcolor=mygray, innertopmargin=2pt, innerbottommargin=5pt, skipabove=10pt, skipbelow=5pt]{mdframed}
% \newcommand{\theoremconfig}{backgroundcolor=mygray,innertopmargin=2pt}

\renewenvironment{proof}
{ 
    \vspace{6 pt}
    \begin{mdframed}[backgroundcolor=white, skipabove=0pt, skipbelow=0pt, innertopmargin=0pt, innerbottommargin=0pt, bottomline=false,topline=false,rightline=false]%
    \noindent \textit{proof.}  
}
{%
    \qed
    \end{mdframed}
    \vspace{5 pt}
}

\newmdtheoremenv[]{theorem}{Theorem}[chapter]
\newmdtheoremenv{corollary}[theorem]{Corollary}
\newmdtheoremenv{proposition}[theorem]{Proposition}
\newmdtheoremenv{lemma}[theorem]{Lemma}
\newmdtheoremenv{conjecture}[theorem]{Conjecture}

\theoremstyle{definition}
\newmdtheoremenv{definition}[theorem]{Definition}
\newmdtheoremenv{example}[theorem]{Example}
\newmdtheoremenv[backgroundcolor = blue!5]{note}[theorem]{Note}
 
\theoremstyle{remark}
\newmdtheoremenv[innertopmargin=7pt, skipabove=10pt]{remark}[theorem]{Remark}

\newcommand{\mybox}[1]{
\begin{empheq}
    
\end{empheq}
}

\newcommand\iu{\mathrm{i}\mkern1mu}
\newcommand{\euler}{\mathrm{e}}
\renewcommand{\d}{\mathrm{d}}

\newcommand{\SIR}{\mathrm{SIR}}
\newcommand{\TX}{\mathrm{TX}}
\newcommand{\RX}{\mathrm{RX}}

\newcommand\sep{\textbf{--}}

\newcommand\C{{\mathbb C}}
\newcommand\N{{\mathbb N}}
\newcommand\R{{\mathbb R}}
\newcommand\Z{{\mathbb Z}}
\renewcommand\P{{\mathbb P}}
\newcommand\F{{\mathbb F}}
\newcommand\K{{\mathbb K}}
\newcommand\Q{{\mathbb Q}}
\newcommand\E{{\mathbb E}}
\newcommand\T{{\mathbb T}}
\renewcommand\S{{\mathbb S}}
\newcommand\ind{\mathds{1}}
\renewcommand\cal{\mathcal}

\newcommand{\pli}[1]{\textcolor{blue}{#1}}

